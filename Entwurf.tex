\documentclass[a4paper,12pt]{report}
\usepackage[a4paper]{geometry}
\usepackage[ngerman]{babel}
\usepackage{fontspec}
\setmainfont[Ligatures=TeX]{Linux Libertine O}
\usepackage{csquotes}
\usepackage{hyperref}
\title{Hamelner Party-Broker Warenwirtschaftssystem}
\author{Florian Bussmann \and Leon Westhof \and Jona Stubbe}
\begin{document}
% \frontmatter
\maketitle
\tableofcontents
% \mainmatter
\part{Klassendiagramm}
\part{Methodenbeschreibung}
\section{Kunde}
\begin{description}
\item[\{set, get\}\{Name, VorName, ID, Straße, Hausnummer, PLZ, Ort\}]
\item[rueckgabe(posten, zeit, menge)]
\item[verlustMelden(posten, zeit, menge)]
\item[ausleihe(ausleihe)]
\item[abrechnung(zeit) -> RechnungPosten{[]}]
\item[berechneUmsatz() -> geld]
\item[transaktionenSammeln() -> String{[]}]
\item[kaufen(posten, menge)]
\end{description}
\section{Ausleihe}
\begin{description}
\item[buchen()]
\item[getPosten() -> LagerPosten]
\item[getMenge() -> menge]
\item[getZeit() -> zeit]
\item[verlust(menge) -> menge]
annuliert eine bestimmte Menge in der Ausleihe, gibt ggf. die Menge, die von anderen Ausleihen gedeckt wird, zurück
\item[rueckgabe(menge) -> menge]
bucht eine bestimmte Menge zurück, gibt ggf. die Menge, die von anderen Ausleihen gedeckt wird, zurück
\end{description}
\section{LagerPosten}
\begin{description}
\item[toString() -> String]
gibt die Daten des Postens, zB in der Form \enquote{<Bestand> x <ID> <Name> (<Einheit>)} als String zurück
\item[istVerkaeuflich() -> boolean]
\item[istVerleihbar() -> boolean]
\item[bestandAendern(menge)]
ändert den Bestand um die angegebene Menge (positiv oder negativ) \\
Exception wenn Bestand negativ werden würde
\item[getBestand() -> geld]
\item[ausleihePreis(zeit, menge) -> geld]
\item[kaufPreis(menge) -> geld]
\item[verlustGebuer(zeit, menge) -> geld]
\end{description}
\section{Betrieb}
\begin{description}
\item[main(args)]
Hauptmenü
\item[bestandAuflisten(modus)]
listet den Bestand, gefiltert nach Kriterien, die durch modus bestimmt werden (Gesamtbestand, verf. Verkaufsware, verf. Verleihware, oÄ), auf
\item[verkaufen()]
der Verkaufs-Menüpunkt
\item[verleih()]
der Verleih-Menüpunkt
\item[abrechnung(kundenID)]
rechnet alle ausstehenden Rechnungspunkte beim Kunden ab
\item[rückgabe()]
der Rückgabe-Menüpunkt
\item[verlust()]
der Verlust-Meldungs-Menüpunkt
\item[umsatzBericht()]
listet die Umsätze der Kunden auf
\item[datenAendern(kundenID)]
ändert die Daten eines Kunden (Menü)
\item[transaktionen(kundenID)]
zeigt die Transaktionen eines Kunden an
\item[anDerUhrDrehen(zeit)]
schreitet die Zeit voran
\end{description}
\part{Testfälle}
\begin{enumerate}
\item
Auflistung des Gesamtbestandes zum Kaufen und zum Verleihen 
\item
Auflistung des aktuell verfügbaren Bestandes zum Ausleihen und zum Kaufen
\item
Kaufen und Verleihen von Mengen, die größer sind als die verfügbaren Mengen(sollte nicht möglich sein)
\item
Rückgabe von Mengen, die die Ausleihe übersteigen(sollte nicht möglich sein)
\item
Rückgabe testen(auch unvollständig)
\item
Abrechnung testen
\item
Umsatzliste testen
\item
Transaktionen auflisten
\item
Kundendaten ändern 
\item
neuen Kunden aufnehmen
\item
Computer-Mensch Dialog überprüfen( und nicht Aktionen auswählen können, die nicht existent sind)
\item
Kauf von nicht verkäuflichen Objekten(sollte nicht möglich sein)
\item
Leih von von nicht verleihbaren Objekten(sollte nicht möglich sein)
\item
Zeitpunkt in die Vergangenheit ändern bzw. negative Zeitdauer(sollte nicht möglich sein)

\end{enumerate}
\appendix
\part{Eigenständigkeitserklärung}
\end{document}
