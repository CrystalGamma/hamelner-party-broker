\documentclass[a4paper,12pt]{article}
\usepackage[a4paper]{geometry}
\usepackage[ngerman]{babel}
\usepackage{fontspec}
\setmainfont[Ligatures=TeX]{Linux Libertine O}
\usepackage{csquotes}
\usepackage{hyperref}
\title{Hamelner Party-Broker Warenwirtschaftssystem}
\author{Florian Bussmann \and Leon Westhof \and Jona Stubbe}
\begin{document}
\maketitle
\tableofcontents
\part{Klassendiagramm}
\part{Methodenbeschreibung}
\section{Kunde}
\section{Ausleihe}
\section{LagerPosten}
\section{Betrieb}
\part{Testfälle}
\begin{enumerate}
\item
Auflistung des Gesamtbestandes zum Kaufen und zum Verleihen 
\item
Auflistung des aktuell verfügbaren Bestandes zum Ausleihen und zum Kaufen
\item
Kaufen und Verleihen von Mengen, die größer sind als die verfügbaren Mengen(sollte nicht möglich sein)
\item
Rückgabe von Mengen, die die Ausleihe übersteigen(sollte nicht möglich sein)
\item
Rückgabe testen(auch unvollständig)
\item
Abrechnung testen
\item
Umsatzliste testen
\item
Transaktionen auflisten
\item
Kundendaten ändern 
\item
neuen Kunden aufnehmen
\item
Computer-Mensch Dialog überprüfen( und nicht Aktionen auswählen können, die nicht existent sind)
\item
Kauf von nicht verkäuflichen Objekten(sollte nicht möglich sein)
\item
Leih von von nicht verleihbaren Objekten(sollte nicht möglich sein)
\item
Zeitpunkt in die Vergangenheit ändern bzw. negative Zeitdauer(sollte nicht möglich sein)

\end{enumerate}
\part{Eigenständigkeitserklärung}
\end{document}
