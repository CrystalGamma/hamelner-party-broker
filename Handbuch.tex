\documentclass[a4paper,12pt,titlepage]{article}
\usepackage[a4paper]{geometry}
\usepackage[ngerman]{babel}
\usepackage{fontspec}
\setmainfont[Ligatures=TeX]{Linux Libertine O}
\newcommand\enquote[1]{{\ttfamily \bfseries #1}}
\usepackage{hyperref}
% \usepackage{graphicx}
\title{Benutzerhandbuch}
\author{Florian Bussmann \and Leon Westhof \and Jona Stubbe}
\begin{document}
\maketitle
\tableofcontents

\part{der Startbildschirm}
Nach dem Programmstart sehen Sie in der Ausgabe \enquote{Kunde Mertens, Robert erfolgreich erstellt. Seine ID ist 1.}.
Diese Standartmeldung sagt Ihnen, dass ein Kunde erstellt wurde mit der Kunden ID 1 und dem Namen Robert Mertens.

In der zweiten Zeile steht \enquote{Für Befehlsliste 0 eingeben}.
Dieser Hinweis sagt Ihnen, dass Sie in der Befehlszeile die 0 für die Liste aller möglichen Befehle eingeben können.
Sie können als vorgebildeter nutzter aber auch jede andere Nummer, die sich von anderen Nutzungen her kennen eingeben.

Nachdem Sie die 0 eingegeben haben, sehen Sie in den ersten 3 Zeilen den aktuellen Kunden mit dem Sie nun arbeiten.
Sollte dieser nicht neben Ihnen am Arbeitstisch sitzten oder sie jemand anderen Bearbeiten wollen, lesen Sie im Kapitel Kunden wechseln nach wie das geht.

\part{Die Befehle}
Jeden eingegebenen Befehl müssen Sie mit Enter bestätigen.

\section{Der Status}
Mit dem Befehl 0 erhalten Sie einige Zeilen, die den aktuell ausgewählten Kunden beschreiben,
gefolgt von der Befehlsliste.

\section{Liste aller zur Zeit verkäuflichen Objekte}
Mit 1 können Sie im ersten Teil alle Objekte sehen,
 die einerseits verkäuflich sind und andererseits zur Zeit verfügabr also im Lager sind und im zweiten Teil der Liste sehen Sie alle Dienstleistungen.

In der ersten Zeile sehen sie die Meldung \enquote{Auflistung der aktuell verfügbaren verkäuflichen Gegenstände}
 und in der Zweiten die Überschriften der folgenden Tabelle: ID und Artikel.
Vor den Objekten sehen Sie die ID der Objekte, welche immer gleich ist
 und auf dieses Objekt kennzeichnet.
Damit können Sie in weiteren Funktionen diese Objektgruppe auswählen,
 jedoch erhalten Sie immer auch eine Liste mit den Objekten auf welche Sie diese Funktionen anwenden können.

Dahinter sehen Sie die aktuelle verfügbare Menge an Objekten und dahinter den Namen des Objektes.
Hinter den Namen sehen Sie in Klammern die entsprechenden Kosten für das Objekte.
Bei Objekten sehen Sie zuerst den Kaufpreis, dann die Handlinggebühr und die tägliche Leihgebühr.
Dienstleistungen können Sie an dem unterschiedlichen Formatierungsstil erkennen:
Bei diesen stehen einfach die Kosten für eine Nutzung der Dienstleistung dahinter.

\section{Liste aller zur Zeit verleihbaren Gegenstände}
Mit 2 in der Eingabezeile können Sie im alle Objekte sehen,
 die einerseits verleihbar sind und andererseits zur zeit verfügbar, also im Lager sind.

In der ersten Zeile sehen sie die Meldung \enquote{Auflistung der aktuell verfügbaren verkäuflichen Gegenstände}
 und in der Zweiten die Überschriften der folgenden Tabelle: ID und Artikel.
Vor den Objekten sehen Sie die ID der Objekte, welche immer gleich ist und auf dieses Objekt kennzeichnet.
Damit können Sie in weiteren Funktionen diese Objektgruppe auswählen,
 jedoch erhalten Sie immer auch eine Liste mit den Objekten auf welche Sie diese Funktionen anwenden können.

Dahinter sehen Sie die aktuelle verfügbare Menge an Objekten und dahinter den Namen des Objektes.
Hinter den Namen sehen Sie in Klammern, die entsprechenden Kosten für das Objekte.
Bei Objekten, die sowohl verleihbar als auch verkäufliche sind sehen sie zuerst den Kaufpreis.
Diese ist hier nicht von Bedeutung.
Danach -- oder bei Objekten, die nur die nur verleihbar sind, am Anfang --
steht die Handlinggebühr und die tägliche Leihgebühr.

\section{Auflistung des verkäuflichen Gesamtbestandes}
Mit dieser Option, welche Sie mit 3 aufrufen, sehen Sie den Bestand an Objekten, die verkäuflich sind.
In der ersten Zeile sehen sie die Meldung \enquote{Auflistung des gesamten verkäuflichen Bestands}
 und in der Zweiten die Überschriften der folgenden Tabelle: ID und Artikel.

Folgend sehen Sie nun die schon bekannte Objekt-ID und den entsprechenden Artikel.

\section{Verkauf eines Objektes}
Mit 4 öffnen Sie ein Menüpunkt zum Verkaufen eines Objektes bzw. zum buchen einer Dienstleistung.
Seien Sie noch mal daraufhin gewiesen, dass alle Aktionen mit dem ausgewählten Konto durchgeführt werden.

In der ersten Zeile sehen sie die Meldung \enquote{Verkauf eines verfügbaren Objekts}
 und in der Zweiten die Meldung \enquote{Zum Verkauf stehen derzeit folgende Produkte zur Verfügung:}.
Danach folgen die Überschriften der folgenden Tabelle: ID und Artikel.

Vor den Objekten sehen Sie die ID der Objekte, welche immer gleich ist und auf dieses Objekt kennzeichnet.
Damit können Sie in weiteren Funktionen diese Objektgruppe auswählen,
 jedoch erhalten Sie immer auch eine Liste mit den Objekten auf welche Sie diese Funktionen anwenden können.

Dahinter sehen Sie die aktuelle verfügbare Menge an Objekten und dahinter den Namen des Objektes.
Hinter den Namen sehen Sie in Klammern, die entsprechenden Kosten für das Objekte.
Bei Objekten, die sowohl verleihbar als auch verkäufliche sind sehen sie zuerst den Kaufpreis.
Danach -- oder bei Objekten, die nur die nur verleihbar sind, am Anfang --
steht die Handlinggebühr und die tägliche Leihgebühr.

In der letzten Zeile sehen sie die Meldung \enquote{Welches Produkt möchten Sie verkaufen?}.
Dahinter schrieben Sie einfach die Produkt-ID. Bei einer Falscheingabe erhalten Sie die Meldung \enquote{Produktnummer ungültig.}.
Nachdem Sie nun ein Objekt gewählt haben,
 erhalten Sie bei Objekten eine Auskunft über die verfügbare Menge und in Klammer die Kosten.

Danach werden Sie nach der Menge gefragt, die Sie ausleihen möchten.
Bei Dienstleistungen werden Sie gefragt wie viele Sie benötigen.
Nach dieser Eingabe erhalten Sie eine Zusammenfassung des Gewählten mit Kosten. 
Diese können Sie nun mit j bestätigen, wobei Sie eine Bestätigung der Buchung erhalten oder Abbrechen,
 wobei die Meldung \enquote{Verkauf abgebrochen.} ausgegeben wird.

\section{verleihbarer Gesamtbestand}
Mit dieser Option, welche Sie mit 5 aufrufen, sehen Sie den Bestand an Objekten,
 die verkäuflich sind und alle angebotenen Dienstleistungen.

In der ersten Zeile sehen sie die Meldung \enquote{Auflistung des gesamten verkäuflichen Bestands}
 und in der Zweiten die Überschriften der folgenden Tabelle: ID und Artikel.
Folgend sehen Sie nun die schon bekannte Objekt-ID und den entsprechenden Artikel.

\section{Ausleihe}
Mit 6 öffnen Sie den Menüpunkt zum Verleihen eines Objektes.
Seien Sie noch mal daraufhin gewiesen, dass alle Aktionen mit dem ausgewählten Konto durchgeführt werden.

In der ersten Zeile sehen sie die Meldung \enquote{Ausleihe eines verfügbaren Objekts für eine prognostizierte Zeitdauer}
 und in der Zweiten die Meldung \enquote{Zum Verleih stehen derzeit folgende Produkte zur Verfügung:}.
Danach folgen die Überschriften der folgenden Tabelle: ID und Artikel.
Vor den Objekten sehen Sie die ID der Objekte, welche immer gleich ist und auf dieses Objekt kennzeichnet. 
Dahinter sehen Sie die aktuelle verfügbare Menge an Objekten und dahinter den Namen des Objektes.

Hinter den Namen sehen sie in Klammern, die entsprechenden Kosten für das Objekte.
Bei Objekten sehen sie zuerst den Kaufpreis, dann die Handlinggebühr und die tägliche Leihgebühr.

In der letzten Zeile sehen sie die Meldung \enquote{Welches Produkt möchten Sie verkaufen?}.
Dahinter schrieben Sie einfach die Produkt-ID.
Bei einer Falscheingabe erhalten Sie die Meldung \enquote{Produktnummer ungültig.}.
Nachdem Sie nun ein Objekt gewählt haben,
 erhalten Sie bei Objekten eine Auskunft über die verfügbare Menge und in Klammer die Kosten.
Danach werden Sie nach der Menge gefragt, die Sie ausleihen möchten und abschließend nach der Ausleihdauern in Tagen.
Nach dieser Eingabe erhalten Sie eine Zusammenfassung des Gewählten mit Kosten.

Diese können Sie nun mit j bestätigen, wobei Sie eine Bestätigung der Buchung erhalten oder Abbrechen,
 wobei die Meldung \enquote{Verkauf abgebrochen.} ausgegeben wird.

\section{Rückgabe eines Objektes}
Mit 7 können Sie Objekte zurückgeben.
Seien Sie noch mal daraufhin gewiesen, dass alle Aktionen mit dem ausgewählten Konto durchgeführt werden.

In der ersten Zeile sehen sie die Meldung \enquote{Rückgabe eines entliehenen Objekts}
 und in der Zweiten die Überschriften der folgenden Tabelle: ID und Artikel.
Vor den Objekten sehen Sie die ID der Objekte, welche immer gleich ist und auf dieses Objekt kennzeichnet.
Damit können Sie in weiteren Funktionen diese Objektgruppe auswählen,
 jedoch erhalten Sie immer auch eine Liste mit den Objekten auf welche Sie diese Funktionen anwenden können.
Dahinter sehen Sie die aktuelle verfügbare Menge an Objekten und dahinter den Namen des Objektes.
Hinter den Namen sehen Sie in Klammern, die entsprechenden Kosten für das Objekte.
Bei Objekten sehen Sie zuerst den Kaufpreis, dann die Handlinggebühr und die tägliche Leihgebühr.

Nachdem Sie eines ausgewählt  haben oder bei Falscheingabe zur erneuten Eingabe aufgefordert werden,
 werden sie gebeten die Rückgabemenge anzugeben.
Bei Falscheingabe werden Sie auch hier wieder zu einer erneuten Eingabe aufgefordert.
Auf Mengenfehler werden sie jedoch erst nach der Bestätigung hingewiesen.
In einem solchen Fall wird der Vorgang mit Fehlermeldung abgebrochen.
Ihre Eingabe können Sie nun mit j bestätigen,
 wobei Sie eine Bestätigung der Buchung erhalten oder Abbrechen,
 wobei die Meldung \enquote{Verkauf abgebrochen.} ausgegeben wird.

\section{Verlustmeldung}
Mit 8 können Sie Verluste melden.
Seien Sie noch mal daraufhin gewiesen, dass alle Aktionen mit dem ausgewählten Konto durchgeführt werden.

In der ersten Zeile sehen Sie die Meldung \enquote{Verlustmeldung eines entliehenen Objekts}
 und in der Zweiten die Überschriften der folgenden Tabelle: ID und Artikel.
Vor den Objekten sehen Sie die ID der Objekte, welche immer gleich ist und auf dieses Objekt kennzeichnet.
Damit können Sie in weiteren Funktionen diese Objektgruppe auswählen,
jedoch erhalten Sie immer auch eine Liste mit den Objekten auf welche Sie diese Funktionen anwenden können.
Dahinter sehen Sie die aktuelle verfügbare Menge an Objekten und dahinter den Namen des Objektes.
Hinter den Namen sehen sie in Klammern, die entsprechenden Kosten für das Objekte.
Bei Objekten sehen sie zuerst den Kaufpreis, dann die Handlinggebühr und die tägliche Leihgebühr.

Nachdem Sie eines ausgewählt  haben oder bei Falscheingabe zur erneuten Eingabe aufgefordert werden,
 werden sie gebeten die Verlustmenge anzugeben.
Bei Falscheingabe werden Sie auch hier wieder zu einer erneuten Eingabe aufgefordert.
Auf Mengenfehler werden sie jedoch erst nach der Bestätigung hingewiesen.
In einem solchen Fall wird der Vorgang mit Fehlermeldung abgebrochen.
Ihre Eingabe können Sie nun mit j bestätigen, wobei Sie eine Bestätigung der Buchung erhalten oder Abbrechen,
 wobei die Meldung \enquote{Verkauf abgebrochen.} ausgegeben wird.

\section{Abrechnungen}
Mit dieser Methode, welche mit 9 aufgerufen wird, können Sie die letzten Aktionen des Kunden abrechnen.
In der ersten Zeile sehen Sie die Meldung \enquote{Abrechnung des aktuellen Kunden} und danach die entsprechenden Abrechnungen.

\section{Transaktionen aller Kunden}
Mit 10 wird Ihnen eine Liste  gezeigt in der alle Kunden mit Ihren Gesamtumsatz aufgelistet werden.
In der ersten Zeile sehen Sie die Meldung \enquote{Auflistung aller Kunden mit ihren Umsätzen}.
Folgend sehen Sie alle Kunden beginnend mit Ihrer Kunden-ID, danach ihrem Namen und ihrem Gesamtumsatz.

\section{Transaktionen eines Kunden}
Mit dieser Methode, welche mit 11 aufgerufen wird, können Sie alle Transaktionen des Kunden sehen.
In der ersten Zeile sehen Sie die Meldung \enquote{Auflistung der mit dem aktuellen Kunden durchgeführten Transaktionen}
 und danach die entsprechenden Abrechnungen.

\section{Neuen Kunden Anlegen}
Mit 12 können Sie einen neuen Kunde anlegen.
Sie werden in Folge nach den folgenden Angaben gefragt: Nachnamen,  Vornamen,  Straße,  Hausnummer, Postleitzahl, Ort.
Abschließend erhalten Sie eine Meldung über den angelegten Kunden und die zugewiesene ID.
Achtung: Es wurde nun automatisch der aktive Kunde auf den Neuen gewechselt.

\section{Kundendaten ändern}
Mit dieser Option, welche sie mit 13 aufrufen, können Sie bestimmte Kundendaten ändern.

Sofern sie 1 wählen können sie ihrer komplette Anschrift ändern.
Dabei werden Sie in folge nach Ort, Postleitzahl, Straße und Hausnummer gefragt.
Bei Nummerfalscheingaben erhalten sie eine Fehlermeldung und werden zur Neueingabe aller Parameter aufgefordert.

Wenn sie 2 gewählt haben, können Sie ihre hausnummer und Straße ändern.
Dafür werden Sie zur eingabe Ihrer Straße und dann ihrer Hausnummer gefragt.
Bei Nummerfalscheingaben erhalten Sie eine Fehlermeldung und werden zur Neueingabe aller Parameter aufgefordert.

Mit 3 können sie Ihren Nachnamen ändern. Diesen können sie nun einfach eingeben.

Mit 4 brechen Sie ab und kommen zum Hauptmenü zurück.

\section{Kunden wechseln}
Mit 14 können Sie den Kunden wechseln.
In der ersten zeile erscheint die meldung \enquote{Aktuellen Kunden wechseln}.
Schreiben Sie dazu die Kunden-ID hinter die Meldung \enquote{Geben Sie die Kunden-ID ein:}.
Danach erhalten Sie eine Bestätigung und sind wieder im Hauptmenü.

\section{Zeit ändern}
Mit dieser Option(Nummer 15) können Sie die Zeit ändern.
Es erfolgt die Meldung \enquote{An der Uhr drehen} und danach können Sie die Anzahl Stunden,
 die nun vergangen sein sollen, nach die Frage stellen.
Bei richtiger Eingabe erhalten Sie eine Bestätigung,
 während sie bei falscheingabe nach einer fehlermeldung zurück im Hauptmenü sind.

\section{Beenden}
Mit dieser Option (Nummer 16) können Sie das Programm beenden.
  

\end{document}

